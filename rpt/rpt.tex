\documentclass{article}
\usepackage{graphicx}
\usepackage{plantuml}
\usepackage{biblatex}
\usepackage{listings}

\addbibresource{biblatex-examples.bib}

\AtBeginDocument{%
  \DeclareFontShape{TU}{lmr}{m}{scit}{<->ssub*lmr/m/scsl}{}%
}

\begin{document}

\title{Parallel Systems Project 1}
\author{Karthik Subramanyam CHAKKA}
\date{}

\maketitle

\section{Introduction}
The problem statement is formulated as follows:
Given a $m \times n$ matrix which is sorted in decreasing order both row-wise and column-wise, return the number of negative numbers in grid.

\begin{center}
x = [[4,3,2,-1],[3,2,1,-1],[1,1,-1,-2],[-1,-1,-2,-3]]    
\end{center}
\begin{center}
f(x) = 8
\end{center}

We implement the function $f$ which counts the number of negative numbers in $m \times n$ matrix using methods below:
\begin{itemize}
    \item Iterative
    \item Divide and Conquer
    \item Controlled Recursion
\end{itemize}

In approach we evaluate performance of algorithms whose nature is:
\begin{itemize}
    \item Sequential
    \item Parallel
    \item Hybrid
\end{itemize}

Finally we take advantage of the fact that the matrix is arranged in decreasing order and propose an optimization that can be applied to all the aforementioned algorithms.

\section{Implementation}


\subsection{Iterative}

This is a brute force approach where each element in the matrix is checked 

\begin{lstlisting}[caption = Iterative]
fn seq_iter_1d(v: &[i32]) -> i32{
    v.iter().filter(|&e| *e < 0).count() as i32
}

pub fn seq_iter_2d(v: &[Vec<i32>]) -> i32{
    v.iter().map(| e| seq_iter_1d(e)).sum()
}

\end{lstlisting}

\subsubsection{Sequential}

\subsubsection{Parallel}

\subsubsection{Hybrid}

\subsection{Divide and Conquer}

\subsubsection{Sequential}

\subsubsection{Parallel}

\subsubsection{Hybrid}

\subsection{Controlled Recursion}

\subsubsection{Sequential}

\subsubsection{Parallel}

\subsubsection{Hybrid}

\subsection{Optimized}

\section{Observations}

\section{Conclusion}
Write your conclusion here.

\nocite{*}
% \printbibliography
\end{document}